\chapter{lit review}

\section{introduction}
This research project brings together various disciplines of research. by combining techniques from computer vision with IMU data etc we can build a data capture system that can 

\section{Human Motion and Gait}


\section{Computer Vision} 
\subsection{Computer Vision in robotics}
\subsection{New Perspectives from Animal Borne Cameras}
In large this researched project was inspired by work done in the Mechatronics Lab at the University of Cape Town. In 2017, Patel et al.  \cite{patel2017trackingieee} showed that using animal borne cameras and motion sensors the tail kinematics of the cheetah (Acinonyx Jubatus) could be tracked. Patel's work was partly inspired by Kane et al.;\cite{kane2014falcons} where falcon (Falco Peregrinus) borne cameras were used to better understand airborne pursuit of prey.
\subsection{Human Motion Analysis Using Computer Vision}


\section{Inertial Measurement Units}
\subsection{Inertial Measurement Units in robotics}
\subsection{}
\subsection{Human Motion Analysis Using Inertial Measurement Units}


\section{Mathematical Modelling}
\subsection{math model of the human gait}
\subsection{linear kinematics}
\subsection{rotational matrices}
\subsection{KF and EKF}


\section{Observing Natural Solutions for Robotic Shortcomings}
Naturally the question arises:  why would we want to better understand the dynamics of animals? A persistent problem in the field of modern robotics is that of mobility; robots struggle to navigate real world surfaces and obstacles. Work by Patel et al. \cite{patel2013rapid} shows how we can look towards nature for inspiration to solve this mobility problem.

This follows the central philosophy of bio-inspired robotics as defined by 

As demonstrated by various prototype robots built by Boston Dynamics bipedal robots are severely limited in manoeuvrability when compared to   


\section{conclusion}






