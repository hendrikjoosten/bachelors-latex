\chapter{Introduction}

\section{Background to the study}
Human motion capture systems are often very costly and confine the capture area to a certain space. This introduces difficulties when trying to understand the motion of humans in complex environments.

A recent methodology \cite{patel2017trackingieee} used by the Mechatronics Lab at the University of Cape Town showed data capture with subject-borne cameras and an IMU has shown the viability of these systems.
 

The field of bio-inspired robotics aims to understand various natural phenomena and incorporate these techniques into the design of modern robotics. 

current information surrounding the issue 


previous studies on the issue

and relevant history on the issue
  
\section{Objectives of this study}
This research project aims to show that subject-borne sensors, primarily cameras and IMUs, can provide researchers in the field of biomechanics and bio-inspired robotics with extensive datasets to understand and model the seemingly magical natural world. 



\section{Scope and Limitations}
The research presented herein does not seek to push the boundaries of modern sensor technology, nor does it wish to re-imagine understood and accepted models of natural phenomena. Instead, a methodology is proposed that brings together systems from exciting disciplines of research such that richer datasets can be generated and studied.

It should therefore be understood that the following work serves as a proof of concept and not as a final design of a motion capture system.  
  

  
\section{Plan of development}
The following chapter contains an extensive literature review where various methods of modelling and verifying the human gait has been discussed. There are also sections dedicated to subject borne data capture, computer vision, inertial measurement units (motion sensors), humanoid robotics and mathematical modelling.
  


\section{Report Outline}


