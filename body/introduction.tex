\chapter{Introduction}

\section{Background to the study}
Human motion capture systems are often very costly and confine the capture area to a certain confined space. These limitations prevent us from understanding bipedal motion in complex environments, knowledge that proves to be critical in the development of humanoid robotics. These limitations can be seen in \cite{sandau2014markerless} where 8 cameras and stereo vision was used to recreate a 3D model of a walking person. 

Recent work \cite{patel2017trackingieee} completed by the Mechatronics Lab at the University of Cape Town showed data capture with subject-borne cameras and sensors can be used to better understand unconstrained movement in a natural environment. The presented work showed the successful kinematic modelling of a cheetah (Acinonyx jubatus) tail whilst running freely. This work was inspired by \cite{patel2014rapid} where the importance of a tail for manoeuvrability was demonstrated.
 
The field of bio-inspired robotics aims to understand various natural phenomena and incorporate these techniques into the design of modern robotics. 


This project seems to be somewhat novel as no research modelling the human gate with subject borne cameras was available at the time of writing. 


  
\section{Objectives of this study}
Depth imagery in the field of human motion capture has been extensively reviewed in \cite{chen2013survey}, where the lack of data from complex movements in different environments is listed as a challenge. This reaffirms the difficulty stated in the previous section.  Solely relying on motion sensors to understand the gait has been reviewed by \cite{picerno201725}. Although this approach was found to be accurate for external environments it has limitation with respect to cost and sensor disturbance. From these reviews it is clear that a middle ground must exist that can combine the strengths of the approaches to provide a holistic solution.

This research project aims to show that subject-borne sensors, primarily a combination of cameras and IMUs, can provide researchers in the field of biomechanics and bio-inspired robotics with extensive datasets to better understand and model the bipedal motion of humans. 

\section{Scope and Limitations}
The scope of this research is to model and estimate the human lower limbs during a flat ground run. This is the first logical step in the iterative design process to eventually understand movement in complex environments.

The research presented herein does not seek to push the boundaries of modern sensor technology, nor does it wish to re-imagine understood and accepted models of natural phenomena. Instead, a methodology is proposed that brings together systems from exciting disciplines of research such that richer datasets can be generated and studied.

It should therefore be understood that the following work serves as a proof of concept and not as a final design of a motion capture system.  
  

  
\section{Plan of Development}
The following chapter contains an extensive literature review where various methods of modelling and verifying the human gait has been discussed. There are also sections dedicated to subject borne data capture, computer vision, inertial measurement units (motion sensors), humanoid robotics and mathematical modelling.

This is followed by a chapter titled methodology that presents the the planning and ideation of the thesis. It serves as a link between the theoretical work presented in the literature review and the engineering approach and application detailed in the chapters that follow it. It lays out a plan and shows how engineering specifications were generated from a generally defined problem. 

The final three chapters that make up the body of this report are titled "Designing the Data Capture System", "Processing the Captured Data" and "Data Fusion and State Estimation" in order of appearance. True to their title they present the process followed to complete the major milestones of the project.

In closing a chapter is dedicated to presenting and discussing the results obtained, followed by the final chapter that draws conclusions from the presented work and makes recommendations on future work. 


 


