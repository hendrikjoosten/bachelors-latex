\chapter{Data Fusion and State Estimation}

This chapter is dedicated the explaining the mathematical methods and models used to fuse data generated by the cameras and IMU.

\begin{table}[!ht]
\centering
\label{statesForEkf}
\begin{tabular}{ll}
State & Description \\
$x_{body}$  	    	&x Position of body w.r.t. the inertial frame\\
$y_{body}$     	 	    &y Position of body w.r.t. the inertial frame\\
$z_{body}$     		    &z Position of body w.r.t. the inertial frame\\
$\theta_{body}$      	&Pitch of body w.r.t. the inertial frame\\
$\phi_{body}$      	    &Roll of body w.r.t. the inertial frame\\
$\psi_{body}$      	    &Yaw of body w.r.t. the inertial frame\\
$\theta_{LH}$     	&Pitch of left thigh w.r.t. left hip\\
$\psi_{LH}$      	&Yaw of left thigh w.r.t. left hip\\
$\theta_{LK}$    	&Pitch of left calf w.r.t. left knee\\
$\theta_{LA}$   	&Pitch of left foot w.r.t. left ankle\\
$\theta_{RH}$    	&Pitch of right thigh w.r.t. right hip\\
$\psi_{RH}$      	&Yaw of right thigh w.r.t. right hip\\
$\theta_{RK}$   	&Pitch of the right calf w.r.t. right knee\\
$\theta_{RA}$  		&Pitch of the right foot w.r.t. the right ankle\\
\end{tabular}
\caption{Table showing the different states of the model to be determined by the kalman filter.}
\end{table}

we will use derivatives

al the derivatives

$$ 					q = [ 
x_{body}        	\; 	    	
y_{body}        	\;    	 	    
z_{body}        	\;    		    
\theta_{body}   	\;   	
\phi_{body}     	\;  	  
\psi_{body}     	\;    
\theta_{LH}    		\;
\psi_{LH}   		\;
\theta_{LK}   		\;
\theta_{LA}   		\;
\theta_{RH}    		\;
\psi_{RH}    		\;
\theta_{RK}   		\;
\theta_{RA}  		
]					$$ 

all the states totalling 42 states

$$	Q =	 [q \; \dot{q} \; \ddot{q}] $$

all 42 and their equations

positional
$$ \ddot{x}_{k+1} = \ddot{x}_{k} + \sigma_{\ddot{x}}^{2} $$
$$ \dot{x}_{k+1} = \dot{x}_{k} + \ddot{x}_{k}T + \sigma_{\dot{x}}^{2} $$
$$ x_{k+1} = x_{k} + \dot{x}_{k}T + \sigma_{x}^{2} $$

angular
$$ \ddot{\theta}_{k+1} = \ddot{\theta}_{k} + \sigma_{\ddot{\theta}}^{2} $$
$$ \dot{\theta}_{k+1} = \dot{\theta}_{k} + \ddot{\theta}_{k}T + \sigma_{\dot{\theta}}^{2} $$
$$ \theta_{k+1} = \theta_{k} + \dot{\theta}_{k}T + \sigma_{\theta}^{2} $$

since all states are either positional(body) or angular(body and limbs)
matrices:
rotational matrices


\begin{center}
x axis
$\begin{bmatrix} 
1 & 0 & 0 \\ 
0 & \cos{\phi} & -\sin{\phi} \\ 
0 & \sin{\phi} & \cos{\phi}  
\end{bmatrix}$

y axis
$\begin{bmatrix} 
\cos{\theta} & 0 & \sin{\theta} \\ 
0 & 1 & 0 \\ 
-\sin{\theta} & 0 & \cos{\theta}  
\end{bmatrix}$

z axis
$\begin{bmatrix} 
\cos{\psi} & -\sin{\psi} & 0 \\ 
\sin{\psi} & \cos{\psi} & 0 \\ 
0 & 0 & 1  
\end{bmatrix}$
\end{center}

solving for the angles

%see paper notes ASWELL

\textbf{front cameras}\\
point 1 right knee\\
point 2 left knee\\
point 3 right foot\\
point 4 left foot\\

\textbf{back cameras}\\
point 1 right calf\\
point 2 left calf\\
point 3 right heel\\
point 4 left heel\\

\textbf{front}\\
right knee
$$ p1xyz = bodyY + bodyZ + R1 * Thigh $$
left knee
$$ p2xyz = bodyY + bodyZ + R1 * Thigh $$
right foot
$$ p3xyz = bodyY + bodyZ + R1 * Thigh + R2 * Calf + R3 * Foot $$
left foot
$$ p4xyz = bodyY + bodyZ + R1 * Thigh + R2 * Calf + R3 * Foot $$

\textbf{back}\\
right calf
$$ p1xyz = bodyY + bodyZ + R1 * Thigh + R2 * 0.5 * Calf $$
left calf
$$ p2xyz = bodyY + bodyZ + R1 * Thigh + R2 * 0.5 * Calf $$
right heel
$$ p2xyz = bodyY + bodyZ + R1 * Thigh + R2 * Calf $$
left heel
$$ p2xyz = bodyY + bodyZ + R1 * Thigh + R2 * Calf $$

\section{Understanding the Data Sources}

It is important to understand the different parameters that mathematically quantify cameras. These parameters can be devided into \textit{extrinsic} and \textit{extrinsic}. Extrinsic camera variables related to the cameras position in the inertial frame and the direction the camera is facing. These can be summarized by the extrinsic camera matrix 

$$[ R \, |\, \boldsymbol{t}] = 
\left[ \begin{array}{ccc|c} 
r_{1,1} & r_{1,2} & r_{1,3} & t_1 \\
r_{2,1} & r_{2,2} & r_{2,3} & t_2 \\
r_{3,1} & r_{3,2} & r_{3,3} & t_3 \\
\end{array} \right]$$



\section{State Estimation}
This section will mathematically explain the Kalman filter and its implementation in this project.

Process equation of the kalam filter.
from states to emasurements
$$ X_{k+1} = FX_{k} + w_{k}  $$

our state, contained in the vector X can be estimated by applying the process matrix F to our current known state. the term w is the noise variable that accounts for process noise.

Measurement equations
from measurements to states.
$$ Y_k = H_{k}X_{k} + v_{k} $$

w will be contained int he matrix Q

while v will be contained in the matrix R1


linearizing nonlinear system we get the EKF


\section{Q Matrix, R Matrix and Initialization}
This section will discuss the final components of the EKF namely the Q matrix containing the various process noise variations, R matrix containing the various measurement noise variances and the initial state values.









