\thispagestyle{plain}
\addcontentsline{toc}{section}{Appendix A: Calibration}
{\Large Appendix A: Calibration}
\vskip 5mm
\hrule
\vskip 5mm


\section*{Vision Calibration}
Before calibrating the stereo cameras and their singular quantifications it is important to take a step back and determine the camera matrix for each individual camera. This allows us to model the cameras mathematically individually. This will allow us to accurately model elements such as lens distortion, principal ray offset and uncertainties accurately. The pinhole camera model is used to quantify camera parameters.  

MATLAB has a built in stereo camera calibration application that can calibrate a set of cameras and create an object containing all the essential parameters, including the individual camera intrinsics and relative camera extrinsic.

To calibrate the cameras pictures of a white and black chequerboard is fed in to the calibration application that then mathematically determines the camera parameters. In order to make this process more efficient a video of the the moving chequerboard was taken and various frames of importance extracted using a simple MATLAB script. The code for this script and the calibrated camera models can be found on the accompanying CD

\section*{Smartphone Calibration}
The smartphone sensor is calibrated during the construction of the smartphone. The hardware detail of the sensor could not be found as the sensor model is not identified. The only correction in smartphone sensor data was to detect the various sensor biases. This was explored in chapter 6 of the report.