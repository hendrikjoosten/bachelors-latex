\pagestyle{plain}

\addcontentsline{toc}{section}{Abstract}

{\Large Abstract}
\vskip 5mm
\hrule
\vskip 5mm

This research aims to extend the work completed by the Mechatronics Lab at the University of Cape town that studied the use of subject borne cameras to analyse, model and estimate the kinematic motion of a cheetah tail. Image processing has progressed significantly within the last few years and studying motion with subject borne cameras allows data capture in unconstrained environments. With this improvements in sensor technology and long standing techniques of data fusion and state estimation new models for complex motion can be created.

This research proposes a methodology for understanding the bipedal motion of humans in natural environments using wearable data capture systems consisting of cameras and sensors. The system is designed using common engineering methods and a prototype constructed using limited resources. 

Testing is then performed and a dataset successfully obtained. This dataset is then processed and applied to a model to better quantify critical elements of the human gait using software tools.   

The proposed system is novel and serves as a proof of concept that can be adapted and improved based on the availability of equipment and end user needs. The research concludes by analysing the methodology and identifying its general strengths and weaknesses, with opportunities for future listed. 

