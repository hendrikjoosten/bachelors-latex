In recognition of the lack of awareness of and available insight into planetary exploration and space science in general, this project involved the design and development of an interactive, scaled-down model of an exploratory rover vehicle, the Mars Curiosity Rover, developed by JPL at the California Institute of Technology. \textit{Curiosity} is the most recent and advanced rover vehicle to explore the surface of Mars and the design of the scaled replication of it aimed to include the mobility and vision capabilities that it has as well as a software system containing a front-end user-interface with which users could operate the model. The model was intended for use in educational environments and the combination of the working model with the front-end application was seen as an opportunity to use the modern and accessible connected technologies of today to create an engaging and insightful experience. With the above mindset, the project was initiated with a review of planetary exploration in general followed by research into \textit{Curiosity} and the Mars Science Laboratory Mission of which it was part. The project followed a standard engineering design structure thereafter which included comprehensive review and analysis of client requirements from which a list of technical specifications was formulated for hardware, electrical and software components of the model. From this point onwards, the entire design was broken down into functional components, each of which underwent conceptual development and detailed design. During the latter stages of the detailed design, the components were reintroduced in a convergent manner to result in a complete and dynamic 3D CAD model of the rover, a design of the electrical system and that of the software systems. Much of the mechanical system of the model involved 3D printing parts in order to achieve a level of realism. The electrical system comprised of a computational subsystem which included the Intel Edison Arduino Breakout board running a distribution of Linux as well as actuation and power modules and components. The software system aimed to draw on popular and impactful open-source projects in the JavaScript and web communities, a proof-of-concept of the use of JavaScript and Node.js in an IoT-embedded context. After review of the detailed designs, the hardware, electrical and software systems were manufactured or developed and finally integrated to result in the final product. The end-result was put through multiple tests closely resembling those that \textit{Curiosity} was put through before the launch to Mars. A verification of the final model against the technical specifications showed that the model was successfully designed in accordance with the client requirements.