\pagestyle{plain}

\addcontentsline{toc}{section}{Abstract}

{\Large Abstract}
\vskip 5mm
\hrule
\vskip 5mm

This research aims to extend work completed by the Mechatronics Research Lab at the University of Cape Town that studied the use of subject borne cameras to analyse, model and estimate the kinematic motion of a cheetah tail. Image processing has progressed significantly within the last few years and studying motion with subject borne cameras allow data capture in unconstrained environments. Improvements in sensor technology and long standing techniques of data fusion and state estimation new models for complex motion can be created.

A methodology for understanding the bipedal motion of humans in natural environments using wearable data capture systems consisting of cameras and sensors is proposed. The system is designed using common engineering methods and a prototype constructed using widely available technologies. Cameras as a sensor to capture motion data have been used on animals in past studies, but at the time of writing no wearable camera based system for human motion capture could be found.  

Testing is performed and a dataset successfully obtained. This dataset is processed and applied to a model by using an Extended Kalman Filter to better quantify critical elements of the human gait. The final results are analysed and discussed, followed by conclusions outlining strengths and drawbacks of the system.

The proposed system is novel and serves as a proof of concept that can be adapted and improved based on the availability of equipment and end user needs. The research is finalized by presenting possible avenues of future work to improve and simplify such a system.

